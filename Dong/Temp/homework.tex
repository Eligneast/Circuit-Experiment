\documentclass[12pt]{article}
\usepackage[heading,fancyhdr]{ctex}
\usepackage{amsmath,amssymb,geometry,lastpage,ulem}
\usepackage{array,tabularx,tabulary,mhchem,xspace}
\usepackage{floatrow,subfig,multirow,bigstrut}
\usepackage{siunitx,booktabs,longtable,graphicx,xfrac}
\lineskiplimit=1pt
\lineskip=3pt
\geometry{
    top=25.4mm, 
    left=25mm, 
    right=25mm, 
    bottom=25mm,
    headsep=5.9mm,
}
\ctexset{
    section = {format+=\raggedright}
}

\title{中国式现代化的特点与实践}
\author{材料科学与工程学院 22301077 张蕴东}
\date{}

\begin{document}

\maketitle

\section*{中国式现代化的特点与实践}

新中国成立以来,在中国共产党的坚强领导下,我国取得了诸多令人瞩目的成就,走出了一条独具特色的社会主义道路。当前,随着国际局势的不断变化和国内情况的不断发展,我国即将踏上新的征程。要在新时代继续走好中国特色社会主义道路,必须深入理解并把握中国式现代化的独特特点。

\subsection*{中国式现代化的特点}

\subsubsection*{1. 人口规模巨大}

我国有14亿多人口,要实现整体现代化,这在人类历史上是前所未有的。如此庞大的人口规模,要求我们付出更多努力,克服更多困难。同时,这也意味着我国的现代化进程将对全球产生广泛影响,为人类进步事业做出更大贡献。与其他国家相比,我国的现代化进程需要在保障每一个公民的基本需求和权利的基础上,推进整体社会的进步。

\subsubsection*{2. 共同富裕}

共同富裕是中国特色社会主义的本质要求。我们不仅要做大蛋糕,还要分好蛋糕。在现代化进程中,必须主动解决地区差距、城乡差距和收入差距问题,扎实推进共同富裕,防止两极分化。只有全体人民共享现代化成果,才能真正实现共同富裕。为此,我们需要在经济发展过程中,加强社会保障和公共服务,确保所有人都能享受到现代化带来的好处。

\subsubsection*{3. 物质文明与精神文明相协调}

中国式现代化不仅关注物质财富的增长,还注重精神文明的建设。我们坚持社会主义核心价值观,加强理想信念教育,弘扬爱国主义、集体主义和英雄主义精神,传承和弘扬中华优秀传统文化。通过这些努力,我们力求在现代化进程中实现物的全面丰富和人的全面发展,推动物质文明与精神文明的协调发展。

\subsubsection*{4. 人与自然和谐共生}

我们注重经济建设与生态文明建设的同步推进,拒绝走先污染后治理的老路,选择节约资源、保护环境、绿色低碳的新型发展道路。为此,我们积极应对全球气候变化,力争2030年前实现碳达峰,2060年前实现碳中和,为全人类的可持续发展做出积极贡献。在推动现代化的过程中,我们始终坚持生态优先、绿色发展的理念,推动形成绿色生产生活方式。

\subsubsection*{5. 和平发展}

中国始终坚持和平共处、互利共赢的基本准则,处理国际关系时坚持多边主义,反对霸权主义和单边主义,积极推动构建人类命运共同体。我们倡导通过和平手段解决国际争端,推动全球治理体系改革和建设,致力于为世界和平与发展贡献中国智慧和中国力量。

\subsection*{实现中国式现代化的关键路径}

\subsubsection*{1. 坚持中国共产党的全面领导}

中国共产党为中国式现代化道路指明了方向。坚持在中国共产党的全面领导下推进中国式现代化,中国式现代化之路才能朝着正确的方向前行,朝着光明的前景前进。坚持中国共产党的领导,才能保证我们奋斗目标的稳定性。目标稳定,不会随意更改,这是我们探索中国式现代化的前提,也是我们的独特优势所在。中国共产党是中国现代化之路上的领导核心所在,在中国共产党的全面领导之下,中国式现代化才有了正确的方向。

\subsubsection*{2. 坚持以人民为中心}

以人民为中心要做到以人为本,要保证人民群众的根本利益,为人民群众谋福利,要站在人民的立场上思考问题,来考虑广大人民群众的发展需求,为人民群众谋福祉。在中国式现代化的发展道路上我们要激发人民群众的主观能动性,凸显人民群众的主体性,让人民群众发挥自身的创造性,使得人民群众积极投身到中国式现代化的实践中去,为中国式现代化添砖加瓦,使得发展成果得以共享。

\subsubsection*{3. 坚持本国国情和现代化发展规律}

中国式现代化不是照搬他国的经验做法,而是在参考他人经验做法的基础之上,探索出符合本国发展的现代化道路。中国式现代化不是简单地发展经济,机械地发展经济建设而忽视其他方面的发展,最后只会导致失败。而把中国式现代化的中国特色转变为成功实践,需要进行全面的发展,其中包括政治、经济、文化等多个方面,我们不仅要保障物质方面的发展还要提高精神方面的发展,不断通过学习理论知识,来武装自己,更好地为中国式现代化增添助力。

\subsubsection*{4. 坚持人与自然和谐共生}

中国式现代化是贯彻新发展理念的现代化,通过学习贯彻新发展理念,做到人与自然和谐共生。在追求发展的同时也不要忽视对环境的保护,不能为了经济发展而肆意破坏环境,要为子孙后代谋福祉。环境问题是关系到全部人类的问题,我们不能为了一己之私,造成无法挽回的后果,要坚持提升生态系统质量,推进可持续发展,加强环境治理。

\subsection*{总结}

中国式现代化是在中国共产党的全面领导下从我国的基本国情出发的,具有鲜明的奋斗目标,不是生搬硬套其他国家的发展模式,是具有鲜明特点的中国式现代化。把中国式现代化的中国特色转变为成功实践,需要我们继续努力,为其他国家的现代化发展提供一定的借鉴意义。通过不断努力和探索,中国式现代化必将在实现中华民族伟大复兴的道路上创造辉煌,为世界现代化发展提供中国方案和中国智慧。

\end{document}
