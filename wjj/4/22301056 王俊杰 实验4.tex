\documentclass[a4paper,utf8]{article}
\usepackage[heading,fancyhdr]{ctex}
\usepackage{amsmath,amssymb,geometry,lastpage,ulem}
\usepackage{array,tabularx,tabulary,mhchem,xspace}
\usepackage{floatrow,subfig,multirow,bigstrut}
\usepackage{siunitx,booktabs,longtable,graphicx,xfrac,nameref}
\lineskiplimit=1pt
\lineskip=3pt
\geometry{
    top=25.4mm, 
    left=25mm, 
    right=25mm, 
    bottom=25mm,
    headsep=5.9mm,
}
\ctexset{
    section = {format+=\raggedright}
}
\newcommand{\fgref}[1]{图~\ref{#1}\xspace}
\newcommand{\seqref}[1]{式~(\ref{#1})}
\newcommand{\expinfo}[7][无]{
    {\zihao{-3}\bfseries\songti
    实验名称:\uline{\hfill\mbox{#2}\hfill} \\[2.9mm]
    学\quad 号:\uline{\makebox[25mm]{#3}}\hfill
    姓\quad 名:\uline{\makebox[25mm]{#4}}\hfill
    班\quad 级:\uline{\makebox[25mm]{#5}} \\[2.9mm]
    合作者:\uline{\makebox[25mm]{#1}} \hfill
    桌\quad 号:\uline{\makebox[25mm]{#6}}\hfill\makebox[25mm+4em]{}\\[2.9mm]
    实验日期:\uline{\makebox[30mm]{#7}}\hfill\mbox{} \\[58.7mm]
    }
}
\newcommand{\pointingbox}{
    {\zihao{4}\bfseries\songti%
    实验考核\\[3mm]
    \extrarowheight=3mm
    \begin{tabularx}{150mm}{|X|X|X|X|X|}\hline
        \hfil 项目 \hfil  & \hfil 实验预习 \hfil & \hfil 实验过程 \hfil & \hfil 分析与讨论 \hfil & \hfil 总评 \hfil \\[3mm] \hline
        \hfil 评价 \hfil &  &  &  &  \\[3mm] \hline
    \end{tabularx}
    }
}
\newcommand{\derivative}[2]{\frac{\mathrm{d} #1}{\mathrm{d} #2}}
\newcommand{\thinking}[2]{\textbf{#1}\\
答:\begin{minipage}[t]{0.85\textwidth}
    #2
\end{minipage}}
\pagestyle{fancy}
\fancyhf{} \fancyhead[C]{电路基础实验} \fancyfoot[C]{\thepage~/~\pageref{LastPage}}
\newcounter{Rownumber}
\newcommand*{\Rown}{\stepcounter{Rownumber}\theRownumber}
\newcommand*{\resetRown}{\setcounter{Rownumber}{0}}
\newcommand{\qrange}[3]{\qtyrange[range-phrase = \text{$\sim$},range-units =single]{#1}{#2}{#3}}
\floatsetup[table]{capposition=top}
\newcolumntype{C}{>{\hfil}X<{\hfil}}
\renewcommand{\Nameref}[1]{\textbf{\ref{#1}~\nameref{#1}}} %导入导言
\newcommand*{\Usa}{\SI{5}{\V}}
\newcommand*{\Usb}{\SI{12}{\V}}
\ctikzset{
    resistors/scale=0.7,
    diodes/scale=0.6}
\begin{document}
\begin{center}
    {\mbox{}\\[7em]\zihao{2}\bfseries\songti%
    电路基础实验报告}\\[34mm]
    \expinfo[王慷]{元件伏安特性的测量}{22301056}{王俊杰}{22 材物}{27}{2024.5.14}
\end{center}
\newpage
\section{实验目的}
\begin{enumerate}
    \item 学习线性电阻元件和非线性电阻元件伏安特性的测试方法。
    \item 学习直流稳压电源、万用表、直流电流表、电压表的使用方法。
\end{enumerate}

\section{实验原理}%简单描述,含必要的公式和附图;
\begin{circuitikz}[american]
    \draw (0,0) to[vsource, l=\Usa, invert] (0,3) to[R=$R_1$] (2,3) to[R=$R_2$] (4,3) to[R=$R_3$] (6,3) to[vsource, l=\Usb] (6,0) -- (0,0);
\end{circuitikz}

\begin{circuitikz}[american]
    \draw (0,0) to[vsource, l=\Usa, invert] (0,3) to[R=$R_1$, v<=$U_1$, i^<=$I_1$] (3,3) to[R=$R_3$, v>=$U_3$, i>^=$I_3$] (3,0) -- (0,0);
    \draw (3,0) node[circ]{};
    \draw (3,3) node[circ]{};
    \draw (3,3) to[R=$R_2$, v<=$U_2$, i<^=$I_2$] (6,3) to[vsource, l=\Usb] (6,0) -- (3,0);
\end{circuitikz}

\begin{circuitikz}[american]
    \draw (0,0) to[vsource, l=\Usa] (0,3) to[R=$R_1$, v<=$U_1$, i^<=$I_1$] (3,3) to[R=$R_3$, i<^=$I_3$, v<=$U_3$] (3,0) -- (0,0);
    \draw (3,0) node[circ]{};
    \draw (3,3) node[circ]{};
    \draw (3,3) to[R=$R_2$, i<^=$I_2$, v<=$U_2$] (6,3) -- (6,0) -- (3,0);
\end{circuitikz}

\begin{circuitikz}[american]
    \draw (0,0) to[vsource, l=\Usa, invert] (0,4) to[R=$R_1$, v<=$U_1$, i^<=$I_1$] (3,4) to[R=$R_3$, v>=$U_3$, i>^=$I_3$] (3,2) to[empty diode, v=$U_4$] (3,0) -- (0,0);
    \draw (3,0) node[circ]{};
    \draw (3,4) node[circ]{};
    \draw (3,4) to[R=$R_2$, v<=$U_2$, i<^=$I_2$] (6,4) to[vsource, l=\Usb] (6,0) -- (3,0);
\end{circuitikz}

\section{实验仪表}
    RIGOL DM3058 万用表、RIGOL DP832 直流稳压电源、电路分析实验箱、导线若干。
\section{实验内容及实验数据}
\section{实验结果与分析}
\section{思考题}
\section{实验心得}
\section{原始数据}
\end{document}